
\documentclass[journal,12pt,twocolumn]{IEEEtran}

\usepackage{setspace}
\usepackage{gensymb}

\singlespacing


\usepackage[cmex10]{amsmath}

\usepackage{amsthm}

\usepackage{mathrsfs}
\usepackage{txfonts}
\usepackage{stfloats}
\usepackage{bm}
\usepackage{cite}
\usepackage{cases}
\usepackage{subfig}

\usepackage{longtable}
\usepackage{multirow}

\usepackage{enumitem}
\usepackage{mathtools}
\usepackage{steinmetz}
\usepackage{tikz}
\usepackage{circuitikz}
\usepackage{verbatim}
\usepackage{tfrupee}
\usepackage[breaklinks=true]{hyperref}

\usepackage{tkz-euclide}

\usetikzlibrary{calc,math}
\usepackage{listings}
    \usepackage{color}                                            %%
    \usepackage{array}                                            %%
    \usepackage{longtable}                                        %%
    \usepackage{calc}                                             %%
    \usepackage{multirow}                                         %%
    \usepackage{hhline}                                           %%
    \usepackage{ifthen}                                           %%
    \usepackage{lscape}     
\usepackage{multicol}
\usepackage{chngcntr}

\DeclareMathOperator*{\Res}{Res}
\DeclareMathOperator{\range}{range}

\renewcommand\thesection{\arabic{section}}
\renewcommand\thesubsection{\thesection.\arabic{subsection}}
\renewcommand\thesubsubsection{\thesubsection.\arabic{subsubsection}}

\renewcommand\thesectiondis{\arabic{section}}
\renewcommand\thesubsectiondis{\thesectiondis.\arabic{subsection}}
\renewcommand\thesubsubsectiondis{\thesubsectiondis.\arabic{subsubsection}}


\hyphenation{op-tical net-works semi-conduc-tor}
\def\inputGnumericTable{}                                 %%

\lstset{
%language=C,
frame=single, 
breaklines=true,
columns=fullflexible
}
\begin{document}


\newtheorem{theorem}{Theorem}[section]
\newtheorem{problem}{Problem}
\newtheorem{proposition}{Proposition}[section]
\newtheorem{lemma}{Lemma}[section]
\newtheorem{corollary}[theorem]{Corollary}
\newtheorem{example}{Example}[section]
\newtheorem{definition}[problem]{Definition}

\newcommand{\BEQA}{\begin{eqnarray}}
\newcommand{\EEQA}{\end{eqnarray}}
\newcommand{\define}{\stackrel{\triangle}{=}}
\bibliographystyle{IEEEtran}
\providecommand{\mbf}{\mathbf}
\providecommand{\pr}[1]{\ensuremath{\Pr\left(#1\right)}}
\providecommand{\qfunc}[1]{\ensuremath{Q\left(#1\right)}}
\providecommand{\sbrak}[1]{\ensuremath{{}\left[#1\right]}}
\providecommand{\lsbrak}[1]{\ensuremath{{}\left[#1\right.}}
\providecommand{\rsbrak}[1]{\ensuremath{{}\left.#1\right]}}
\providecommand{\brak}[1]{\ensuremath{\left(#1\right)}}
\providecommand{\lbrak}[1]{\ensuremath{\left(#1\right.}}
\providecommand{\rbrak}[1]{\ensuremath{\left.#1\right)}}
\providecommand{\cbrak}[1]{\ensuremath{\left\{#1\right\}}}
\providecommand{\lcbrak}[1]{\ensuremath{\left\{#1\right.}}
\providecommand{\rcbrak}[1]{\ensuremath{\left.#1\right\}}}
\theoremstyle{remark}
\newtheorem{rem}{Remark}
\newcommand{\sgn}{\mathop{\mathrm{sgn}}}
%\providecommand{\abs}[1]{\left\vert#1\right\vert}
\providecommand{\res}[1]{\Res\displaylimits_{#1}} 
%\providecommand{\norm}[1]{\left\lVert#1\right\rVert}
%\providecommand{\norm}[1]{\lVert#1\rVert}
\providecommand{\mtx}[1]{\mathbf{#1}}
%\providecommand{\mean}[1]{E\left[ #1 \right]}
\providecommand{\fourier}{\overset{\mathcal{F}}{ \rightleftharpoons}}
%\providecommand{\hilbert}{\overset{\mathcal{H}}{ \rightleftharpoons}}
\providecommand{\system}{\overset{\mathcal{H}}{ \longleftrightarrow}}
	%\newcommand{\solution}[2]{\textbf{Solution:}{#1}}
\newcommand{\solution}{\noindent \textbf{Solution: }}
\newcommand{\cosec}{\,\text{cosec}\,}
\providecommand{\dec}[2]{\ensuremath{\overset{#1}{\underset{#2}{\gtrless}}}}
\newcommand{\myvec}[1]{\ensuremath{\begin{pmatrix}#1\end{pmatrix}}}
\newcommand{\mydet}[1]{\ensuremath{\begin{vmatrix}#1\end{vmatrix}}}
\numberwithin{equation}{subsection}
\makeatletter
\@addtoreset{figure}{problem}
\makeatother
\let\StandardTheFigure\thefigure
\let\vec\mathbf
\renewcommand{\thefigure}{\theproblem}
\def\putbox#1#2#3{\makebox[0in][l]{\makebox[#1][l]{}\raisebox{\baselineskip}[0in][0in]{\raisebox{#2}[0in][0in]{#3}}}}
     \def\rightbox#1{\makebox[0in][r]{#1}}
     \def\centbox#1{\makebox[0in]{#1}}
     \def\topbox#1{\raisebox{-\baselineskip}[0in][0in]{#1}}
     \def\midbox#1{\raisebox{-0.5\baselineskip}[0in][0in]{#1}}
\vspace{3cm}
\title{Assignment 19}
\author{Satyam Singh \\ EE20MTECH14015}
\maketitle
\newpage
\bigskip
\renewcommand{\thefigure}{\theenumi}
\renewcommand{\thetable}{\theenumi}
\begin{abstract}
This document explains the g.c.d of polynomial.
\end{abstract}
Download all python codes from 
\begin{lstlisting}
https://github.com/satyam463/Assignment-19/blob/main/Assignment%2019%20.py
\end{lstlisting}
\section{Problem Statement}
Find the g.c.d of each of the following pairs of polynomials.
    \begin{align}
    2x^{5}-x^{3}-3x^{2}-6x+4 , x^{4}+x^{3}-x^{2}-2x-2
    \end{align}
\section{Solution}
Refer Table \ref{table:1}.
\begin{table*}[ht!]
\begin{center}
\begin{tabular}{|l|l|}
\hline
\multicolumn{2}{|c|}{
Let the field be rational numbers}\\[1ex]
\hline
\textbf{Steps} & \textbf{Explanation} \\[0.5ex]
\hline
\text{Say f\brak{x} and g\brak{x} } & 
    \parbox{10cm}{\begin{align}
    f\brak{x}=2x^{5}-x^{3}-3x^{2}-6x+4 \\
     g\brak{x}=x^{4}+x^{3}-x^{2}-2x-2
\end{align}} \\[0.5ex]
\hline
\text{Expanding f\brak{x} in term of g\brak{x}} & 
 \parbox{10cm}{\begin{multline}
 \begin{aligned}
 2x^{5}-x^{3}-3x^{2}-6x+4 = \brak{x^{4}+x^{3}-x^{2}-2x-2}\brak{2x-2}\\+\brak{3x^{3}-x^{2}-6x}
 \end{aligned}
 \end{multline}}\\[0.5ex]
\hline
\text{Expanding degree four polynomial} &
\parbox{10cm}{\begin{multline}
\begin{aligned}
 x^{4}+x^{3}-x^{2}-2x-2 = \brak{3x^{3}-x^{2}-6x}\brak{\frac{1}{3}x+\frac{4}{9}}+\brak{\frac{13}{9}x^{2}+\frac{2}{3}x-2}
 \end{aligned}
 \end{multline}}\\[0.5ex]
\hline
\text{Expanding degree three polynomial} &
\parbox{10cm}{\begin{multline}
\begin{aligned}
 3x^{3}-x^{2}-6x= \brak{\frac{13}{9}x^{2}+\frac{2}{3}x-2}\brak{\frac{27}{13}x-\frac{279}{169}}+\brak{-\frac{126}{169}x-\frac{558}{169}}
 \end{aligned}
 \end{multline}}\\[0.5ex]
 \hline
 \text{Expanding degree two polynomial} &
\parbox{10cm}{\begin{multline}
\begin{aligned}
 \frac{13}{9}x^{2}+\frac{2}{3}x-2 = \brak{-\frac{126}{169}x-\frac{558}{169}}\brak{-\frac{2197}{1134}x+\frac{61009}{7938}}+\brak{\frac{10309}{441}}
 \end{aligned}
 \end{multline}}\\
 & Since it contains scalar polynomial hence the g.c.d of f\brak{x} ,  g\brak{x} is 1.
 \\[0.5ex]
 \hline
\end{tabular}
\caption{Solution}
\label{table:1}
\end{center}
\vspace{-0.5cm}
\end{table*}
\end{document}
